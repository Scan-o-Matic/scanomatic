\documentclass[]{article} % Remove "draft" option in final version!
\usepackage[utf8]{inputenc}
\usepackage{bookman}
\usepackage{amsmath}
\usepackage{amssymb}
\usepackage[a4paper]{geometry}
\usepackage{todonotes}

\newcommand{\TODO}[1]{\todo[inline]{#1}}

\title{A geometrical approach to boids-like AI}
\author{Scientifically Fun}
\date{\today}

\begin{document}
\maketitle


\begin{abstract}
\footnotesize
\noindent  The following exposition relies on the insight (or, in more modest terms, supposition) that the most relevant entities for a movable object -- the fundamental quality that is common to most, if not all, those things which affect members of Connocahetes and Crocodylidae alike -- are \emph{positions}; positions of events, of comrades and prey, of machines and of men.

The method introduced here works by constructing an average of positions of events and externalities, fellows and foes, allowing for simultaneously assigning weights based on distance and category, where the latter (which may also be dependent on distance) is meant to be decided by the behavioural state of the AI. 

The article is concluded with a concrete -- but admittedly tentative -- example of how this method may be employed.
\end{abstract}

\tableofcontents

\thispagestyle{empty}

\newpage

\section{Where do you want to go today?}
\label{sec:introduction}
Well, that depends. What happens where? Is it good or bad? Is it far?

This short essay seeks an answer to the question posed above, by introducing a general systematic model for finding out which way a movable object wants to set its heading at any given instance.

The model presented here is meant for implementation as a part of an artificial intelligence in e.g. a computer game, but there are no fundamental reasons why it could not be adapted for use as a means of alleviating the pains of making the numerous decisions that infest everyday life.
Whether or not this is a good idea is, however, beyond the scope of this article.


\subsection{General events}
\label{sec:general_events}
Assume that an entity capable of reacting to its surroundings -- henceforth called the \emph{Subject} -- is at a given time $t$ is aware of $N_E$ \emph{Events} in its surroundings.
Assign to the Subject and to all Events \emph{positions} in relation to an arbitrary but stationary origin, describing these positions by the vector $\vec{R}_0$ for the Subject, and the vectors $\vec{R}_j$\footnote{of course, $\vec{R}_j$ need not be the \emph{actual} position of the Event, but could instead represent the location of the Event as \emph{perceived} by the Subject; this distinction, however, makes no difference to the implementation}, 
where $j \in \left[1,\, N_E\right] \subset \mathbb{Z}^+$, for the Events.
$N_E$ is then the \emph{number of Events} the Subject may currently be affected by.
Assume then, that the importance of an Event depends on how far away it occurs, describing this by a \emph{distance penalty} function $D(\vec{R}_j -\vec{R}_0)$.
Further, the \emph{importance} of an individual Event is described by assigning to each Event $j$ a weight $w_j$, which may be positive (\emph{attractive}) or negative (\emph{repulsive}).
This is really all that is needed, but for clarity, add a general term $E$ describing the Subject's \emph{conscientiousness} -- how much the Subject cares about Events anyway.

All of the above terms are now combined in the form of a weighted sum, the result of which is a position toward which the Subject feels inclined to move. This sum takes the form

\begin{equation}
 \label{eq:events_raw}
 \vec{E}_0(t) = E_0(t)\sum_{j=1}^{N_{0,E}(t)} w_{0,j}(t) D_0(\vec{R}_j(t) - \vec{R}_0(t))(\vec{R}_j -\vec{R}_0),
\end{equation}

\noindent allowing for time dependence in all parameters. 

In Equation~\eqref{eq:events_raw}, the $_0$ implies that all calculations refer to the Subject. 
Rather than writing $X_0$, the $0$ index will be treated as implied, meaning that it is understood that all parameters may depend on which particular Subject is under consideration.
Likewise the time dependency will not be explicitly written out in the following paragraphs.
It may, however, be a good idea to consider in every step, which parameters need \emph{not} depend on the Subject or the time; fewer variables give results that are easier to understand, and it is in no way to be taken for granted, that more variability will lead to a better model.
For instance, the distance weighting function $D$ could be the same for all entities.\footnote{Reynolds \cite{Reynolds1987} suggests using the true group ``centroid'', while using the distance penalty function $D(\vec{r}) = 1/\left|\vec{r}\right|^{2}$ for the velocities; this particular choice of $D$ can be motivated universally by dimensional arguments: in two dimensions, the number of expected Events within a given distance is proportional to the area, or $\left|\vec{r}_j\right|^2$, leading to the generalisation $D(\vec{r}) = 1/\left|\vec{r}\right|^{d}$ for $d$ dimensions (it should be noted, however, that a similar argument can be made for using $D(\vec{r}) = 1/\left|\vec{r}\right|^{d-1}$, based on the Subject's field of vision)}

To further simplify the notation, rather than always using the difference between vectors explicitly in the mathematical expressions, a \emph{relative} distance vector is defined $\vec{r}_j \equiv \vec{R}_j - \vec{R}_0$, a practice that captures naturally the assumption, that it is the position \emph{in relation} to the Subject that is important, not the ``absolute'' position.

Using these conventions, Equation~\eqref{eq:events_raw} becomes

\begin{equation}
 \label{eq:events}
 \vec{E} = E\sum_{j=1}^{N_E} w_j D(\vec{r}_j)\vec{r}_j.
\end{equation}


\section{An example of categorisation}
\label{sec:categories}
While very general, implementing Equation~\eqref{eq:events} leaves much too many details up to arbitrary decisions, leading -- as hinted at above -- to results that may be hard to interpret.
In the following sections, an example of a subdivision of Events into distinct \emph{Categories} will be presented, in the hope that this will make the implementation less daunting. 
The focus will be primarily on herd behaviour, but the lesson is hopefully general enough to aid in implementing more individually minded Subjects as well.

For clarity, all members of a well defined Category are treated equally.


\subsection{Group psychology}
\label{sec:good_comrades}
How a Subject relates its comrades is probably the most important aspect of any social being.
Comrade positions will therefore be the first Category to be considered, naming it $C$.

Since there may be comrades towards whom the Subject is attracted and others that are near enough already, this Category will be divided into the sub-categories $C_+$ and $C_-$.


\subsubsection{Stick to your comrades\ldots}
\label{sec:bad_comrades}
The most basic property of a herd is that the individuals comprising it stick to their comrades.
Without that, the group will soon lose cohesion, and the term ``herd'' will have lost its meaning.
Imagine, therefore, that a Subject has $N_{C_+}$ comrades it wants to follow.
An effective group position is constructed by summing up the comrades' positions, assigning them weights based on the distance penalty.
The sum is then multiplied by an \emph{urgency factor} $C_+$, based on the subject's current internal state, and then normalised to the number of comrades.\footnote{this normalisation is not strictly necessary, and should be regarded as optional} 
It is assumed that the higher $C_+$ is, the more important the Subject thinks this Category is.

The result is a \emph{point of attraction} $\vec{C}_+$, the direction of which indicates where to the Subject wants to go because of this Category.
The magnitude $|\vec{C}_+|$ relates to how much the Subject wants to follow the group.
$\vec{C}_+$ is expressed as

\begin{equation}
 \label{eq:good_comrades}
 \vec{C}_+ = \frac{C_+}{N_{C_+}} \sum_{j=1}^{N_{C_+}} D(\vec{r}_j) \vec{r}_j.
\end{equation}


\subsubsection{\ldots but don't let 'em step on your toes}
As has been previously observed, see e.g.~\cite{Reynolds1987, Chiang2009}, another important property of natural herd behaviour, is that the individuals do not run into one another.
In his seminal paper of 1987~\cite{Reynolds1987}, Reynolds goes so far as to say that this takes precedence over following the group.
Though this is certainly worth bearing in mind, such minutiae are left up to the implementor to decide, and it will likely depend on the State of the Subject which Category is more important.

Assume there are $N_{C_-}$ comrades that are too close for the Subject's comfort.
Just as with the attractive group, a sum of their positions with respect to the Subject is constructed, where the terms are assigned weights based on the distance penalty function.
The result is multiplied with a factor based on this Category's momentary importance: $C_-$.
Again, the value of $C_-$ will likely be decided by the state of the Subject, however, for clarity it will be assumed to be positive, and the repulsive nature of the point $\vec{C}_-$ taken care of in the next step.
$\vec{C}_-$ is expressed as

\begin{equation}
 \label{eq:bad_comrades}
 \vec{C}_- =  \frac{C_-}{N_{C_-}} \sum_{j=1}^{N_{C_-}} D(\vec{r}_j) \vec{r}_j.
\end{equation}


\subsubsection{All accounted for!}
\label{sec:all_comrades}
Having calculated the vectors $\vec{C}_+$ and $\vec{C}_-$, the total point of attraction for the comrades Category is given by the difference between the two, meaning that the first of the pair acts as an \emph{attractor}, the second as a \emph{repulsor}.

To summarise, the resulting point $\vec{C}$ is written\footnote{note that the vectors are named $\vec{r}_j$ in both sums; this is for convenience only, and does not imply that they are related -- compare this, for instance to name spaces in Python loop structures}

\begin{equation}
 \label{eq:comrades}
 \vec{C} = \vec{C}_+ - \vec{C}_- = \frac{C_+}{N_{C_+}} \sum_{j=1}^{N_{C_+}} D(\vec{r}_j) \vec{r}_j -
                                   \frac{C_-}{N_{C_-}} \sum_{j=1}^{N_{C_-}} D(\vec{r}_j) \vec{r}_j.
\end{equation}


\subsection{Staying in line}
\label{sec:velocities}
Another important aspect of herd behaviour is that the individual Subjects align themselves according to their comrades' directions, and also that they adapt their speed to that of the group.~\cite{Reynolds1987}
This motivates a Category $V$ accounting for velocities and headings of the Subject's comrades.

To start out, define the relative velocity of the Subject and comrade $j$ as $\vec{v}_j \equiv \vec{V}_j - \vec{V}_0$, analogously to how $\vec{r}_j$ was defined in Section~\ref{sec:general_events}.
Just as with the comrades Category, a weighted summation is performed

\begin{equation}
 \label{eq:velocities}
 \vec{V} = \frac{V}{N_V}\sum_{j=1}^{N_V} D(\vec{r_j}) \vec{v}_j,
\end{equation}

\noindent where typically $N_V = N_C = N_{C_+} + N_{C_-}$, though e.g.~\cite{Chiang2009} suggests that a subset of the comrades may be used.
Note that the distance penalty is still calculated using the distance $\vec{r}_j$ to the comrade.


\subsection{Events that are not otherwise identified by name}
\label{sec:noibn}
The number of Categories of Events that could be imagined are, of course, endless.
Some further suggestions are: the landscape, entities other the Subject's Comrades, and scary flying contraptions.
There may also be Events that, simply put, defy categorisation.
These are all summarised here, under a single Category $E$, but the reader is encouraged to envision ways (though not too many!) into which this could comprehensively be subdivided.
Capturing this oversight, this special Category will be borrow it's description from Equation~\eqref{eq:events}.


\subsection{Putting it together}
\label{sec:conclusion}
To conclude, the master equation for the point where to the Subject wants to relocate itself is arrived at by adding the contributions for the different categories

\begin{equation}
 \label{eq:WSOGMM}
 \vec{P} = \vec{E} + \vec{C} + \vec{V}.
\end{equation}


\subsubsection{A hybrid model, and why it may not be a good idea, despite laudable intentions}
\label{sec:hybrid}
Excepting the general Events category, individual weights $w_j$ have not been allowed for this far.
As mentioned above, this is done for clarity, but also for ease of implementation.

Revisiting the example of the comrades Category (Section~\ref{sec:all_comrades}), it may be attractive allowing for the Subject to be differently inclined to move towards, for instance, a panicking comrade, as opposed to one occupied grazing.
This can be accomplished by, for each comrade~$j$, calculating a weight $w_j$,  just as in the general case of Equation~\eqref{eq:events}, transforming Equation~\eqref{eq:comrades} into

\begin{equation}
 \label{eq:comrades_hybrid}
 \vec{C} = \vec{C}_+ - \vec{C}_- = C_+ \sum_{j=1}^{N_{C_+}} w_j D(\vec{r}_j) \vec{r}_j -
                                   C_- \sum_{j=1}^{N_{C_-}} w_j D(\vec{r}_j) \vec{r}_j.
\end{equation}

While this method may bring out nuances of behaviour, caution is urged for two main reasons:
firstly, calculating the individual weights may (depending, of course, on the implementation) be costly; secondly, the increase in algorithmic complexity comes with the threat of an equal decrease in parametric predictability, without any obvious promise of more interesting behaviour.
Of these two reasons, the second is considered the most important by the author. 

It should be stressed that there may well be situations, such as hierarchical groups, where individual weights may be unavoidable, or it may that the herd's behaviour turns out to lack dynamics without it.
The decision will therefore be left for the implementer, with the final advice that they try to start in the simple end.


\section{Points of attraction}
\label{sec:nowwhat}
What to do with the point $\vec{P}$, now that it has been calculated?
This section presents two ideas, first a simple one, based on turning directions, suitable e.g. for Subjects that are restricted to a discrete set of directions, then the outline of an algorithm inspired by molecular physics.


\subsection{Turning and matching velocity}
\label{sec:turning}
The most basic information to be gained from the point of attraction $\vec{P}$ is, perhaps, which way the Subject wishes to turn.
This can be determined via the method outlined in Appendix~\ref{sec:mathematics}.
For instance, the sign of the quantity $B$ derived in the Appendix could be be added to a \emph{turn counter}, and the Subject made to turn once the absolute value of the counter is greater than a specified \emph{turn lag}.

This is a crude method, that needs $\vec{P}$ to be recalculated fairly often, in order to avoid a zigzag trajectory for the Subject in situations when the angle to $\vec{P}$ from the Subject's heading is small.
One way of dealing with this is outlined in Appendix~\ref{sec:turn_urgency}.

Another thing that this method does not take into account is how to alter the speed of the Subject.
Since the magnitude of $\vec{P}$ somehow relates to the desire of the Subject to get there, this value could be used to decide whether an increase or a decrease in speed is required.

A simpler method is to increase the Subject's speed if it is lower than the mean speed of its comrades, and decrease it if their mean speed is lower than that of the Subject, where the mean speed is simply

\begin{equation}
 \label{eq:mean_velocity}
  \left<V\right> = \frac{1}{N_V}\sum_{j=1}^{N_V} \left|\vec{V}_j\right|,
\end{equation}

\noindent using the ``absolute'' velocity vectors $\vec{V}_j$, rather than the relative $\vec{v}_j$.


\section{A physical interpretation}
\label{sec:newton}
If the Category weights are chosen properly, the vector $\vec{P}$ could be interpreted as something akin to a force acting on the Subject.
This suggests an equation of motion for the Subject similar to Newton's Second law:~\cite{Physics}

\begin{equation}
 \label{eq:Newton2}
 \vec{P}=m\vec{a} \Leftrightarrow \frac{\vec{P}}{m} = \vec{a}
\end{equation}

\noindent Here $m$ should not be interpreted as the actual \emph{mass} of the Subject, but rather as the Subject's inertia in a more general sense; its stubbornness perhaps.
It may, however, be a good -- though in no way critical -- idea to make sure that $m$ is of a physically sound order of magnitude, as this will give a sense of the scale of the forces represented by the different categories.

This method may seem farfetched, however, the similarity between boid dynamics and strongly correlated molecular dynamics has often been noted.
The physics of those system involve the complex interplay of close-range repulsive potentials and long-range attractive forces, leading to complicated collective behaviours.
The similarity of this situation to the collision-avoiding and herd-following behaviours of many species of animals has lead to advances in the study and simulation of both fields, as methods from one cross-fertilise the other.~\cite{Eriksson2010}

The simplest and most intuitive way of implementing the equation of motion is probably the Euler method.~\cite{Beta}
If the time step is $dt$, the Euler method gives the velocity and position at time $t$ as

\begin{eqnarray}
 \label{eq:Euler_velocity}
 \vec{V}_t = \vec{V}_{t-dt} + \vec{a}_t dt, \\
 \label{eq:Euler_position}
 \vec{R}_t = \vec{R}_{t-dt} + \vec{V}_t dt. 
\end{eqnarray}

Substituting the acceleration through Equation~\eqref{eq:Newton2}, the Equations~\eqref{eq:Euler_velocity} and~\eqref{eq:Euler_position} can be implemented directly, and augmented with methods to re-scale or cap the velocity, based on the Subject's maximum and minimum speeds, the group speed, etc.
This method also lends itself naturally to implementing velocities and accelerations in physically comprehensible units, which aids the intuition when choosing parameters.


\appendix
\section{Mathematical details}
\label{sec:mathematics}
\subsection{The mathematics of left and right}
\label{sec:left_and_right}
Using projections of vectors, the trigonometric problem of finding a bearing can be reduced to algebraic operations, which are often much cheaper in terms of computational resources, but also easier to deal with when it comes to writing ``bulk processing'' algorithms.
The finer details of this, however, are beyond the scope of this appendix, which will focus on how to find out on which side -- left or right -- of a vector a given point is situated.

To start out, let $\vec{r}_0 = \left(u,\, v\right)$ and $\vec{r}_1 = \left(s,\, t\right)$.
Here, in analogy with the nomenclature used above, $\vec{r}_0$ will belong to the Subject, representing its heading, while $\vec{r}_1$ is an Event vector.
It is the orientation of the latter with respect to the former that is of interest.

The projection of a vector onto another is given by the scalar\footnote{or ``dot''} product~\cite{Beta}

\begin{equation}
\vec{r}_0 \cdot \vec{r}_1 = \left(u,\, v\right) \cdot \left(s, t\right) = u\,s + v\,t = A,
\end{equation}

\noindent where $A$ is a number -- a ``scalar'', as opposed to a vector.
Note that the order of the vectors does not matter.

The quantity $A$ will be positive if $\vec{r}_0$ and $\vec{r}_1$ are pointing along one another, i.e. if the one is oriented within $\pm\pi$ radians of the other.
Conversely, $A$ will be negative if they are pointing counter to one another, meaning that the angle between them is more than $\pm\pi$ radians.
If $A=0$ then $\vec{r}_0$ and $\vec{r}_1$ are exactly orthogonal.

In order to find out on which side of $\vec{r}_0$ the vector $\vec{r}_1$ is pointing, the vector $\vec{r}_0$ will be rotated by $\pi/2$ radians, and the projection of $\vec{r}_1$ on the result calculated.
This rotation is accomplished by simply exchanging the $x$ and $y$ coordinates of $\vec{r}_0$, changing the sign of the latter:\footnote{a general rotation by an angle $\theta$ is performed by multiplying the vector by a rotation matrix from the left: $\vec{r}_\theta = \begin{bmatrix}\cos \theta & -\sin \theta \\ \sin \theta & \cos \theta \end{bmatrix}\vec{r}_0$; see~\cite{Beta}}
$\vec{r}_{0,\perp}=\left(-v,\, u\right)$.
The sought projections is thus

\begin{equation}
 \label{eq:perp_projection}
 \vec{r}_{0,\perp} \cdot \vec{r}_1 = \left(-v,\, u\right) \cdot \left(s, t\right) = -v\,s + u\,t = B.
\end{equation}

The sign of $B$ will determine on which side of $\vec{r}_0$ the point $\vec{r}_1$ is situated:
\begin{itemize}
 \item $B > 0 \Longrightarrow$ $\vec{r}_1$ is to the \emph{left} of $\vec{r}_0$,
 \item $B < 0 \Longrightarrow$ $\vec{r}_1$ is to the \emph{right} of $\vec{r}_0$,
 \item $B = 0 \Longrightarrow$ $\vec{r}_1$ and $\vec{r}_0$ are either \emph{parallel} or \emph{anti-parallel}.
\end{itemize}


\subsection{The computer screen is a different beast}
\label{sec:screen_coordinates}
The results of the Section~\ref{sec:left_and_right} assume a right-handed orthonormal coordinate system, which is the arbitrary convention that is normally used.
The computer screen, however, does not natively employ a right- but a left-handed coordinate system, and hence the result needs to be altered to take this into account.
The effect of this is to change the positive rotation direction from anti-clockwise to clockwise, which changes the meaning of the rotation by $\pi/2$ radians.
For screen coordinates, the main result from Section~\ref{sec:left_and_right} therefore becomes
\begin{itemize}
 \item $B > 0 \Longrightarrow$ $\vec{r}_1$ is to the \emph{right} of $\vec{r}_0$,
 \item $B < 0 \Longrightarrow$ $\vec{r}_1$ is to the \emph{left} of $\vec{r}_0$,
 \item $B = 0 \Longrightarrow$ $\vec{r}_1$ and $\vec{r}_0$ are either \emph{parallel} or \emph{anti-parallel}.
\end{itemize}

Please note that the interpretation of $A$ does not change because of the change of coordinate conventions, as it does not rely on any rotations.


\subsection{A measure of turning urgency}
\label{sec:turn_urgency}
If $\vec{r}_0$ is normalised, the quotient $C$ between $B$ and the length of the vector $\vec{r}_1$ will comply to $C = B/\left|\vec{r}_1\right| \in [-1,\, 1]$.
The more aligned $\vec{r}_0$ and $\vec{r}_1$ are, the closer $C$ will be to $0$, and hence it can be a useful measure of how urgent it is to turn.\footnote{the reader may already have observed that $C$ is simply the sine of the angle between the vectors}
This is tricky, however, because as was noted in Section~\ref{sec:left_and_right}, $B$ does not take into account whether the vectors are parallel or anti-parallel; this information is contained in the quantity $A$.

If the vectors are perfectly anti-parallel, the urgency of turning must be regarded as very high.
One way of ensuring this, is to use the quantity $C$ only if $A>0$, and otherwise use the sign of $B$ as the measure.


\newpage
\bibliography{ai}
\bibliographystyle{unsrt}

% Important! Remove in final version!
%\newpage
%\listoftodos
%\thispagestyle{empty}

\end{document}
